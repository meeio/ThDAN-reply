% Please add the following required packages to your document preamble:
% \usepackage{multirow}

% \newcommand\iw{0.05}
% \newcommand\cw{0.05}
% \renewcommand\tabularxcolumn[1]{m{#1}}
\newcommand\emmax[1]{\textcolor{red}{\textbf{#1}}}
\newcommand\methodyear[1]{\textcolor{blue}{#1}}
\newcommand\Tstrut{\rule{0pt}{2.6ex}}
\newcommand\Bstrut{\rule[-0.9ex]{0pt}{0pt}}

\newcommand{\centeritem}[1]{\noindent\parbox[c]{\hsize}{\footnotesize \vspace*{1mm} #1 \vspace*{1mm}}}

\begin{table*}[htb]
    \renewcommand{\arraystretch}{1.3}
    \caption{ The logical connection between paragraphs of the introduction and sentences of the abstract. }
    \label{table: logical}
    \centering
    \small
    % >{\itshape
    \begin{tabularx}{0.95\textwidth}{c| >{\itshape}X}
        \toprule[0.8pt]
        \textbf{\small Paragraph of Introduction} & \multicolumn{1}{c}{\textbf{\small Sentence of Abstract}}                                                                                                                                                                                                                              \\
        \bottomrule[0.8pt]
        \# 1             & \centeritem{In recent years, many unsupervised domain adaptation (UDA) methods have been proposed to tackle the domain shift problem.}                                                                                                          \\
        \hline
        \# 2             & \centeritem{Most existing UDA methods are derived for Close Set Domain Adaptation (\textit{CSDA}) in which source and target domains are assumed to share the same label space.
            However, target domain may contain unknown class different from the known ones in the source domain in practice, i.e., Open Set Domain Adaptation (\textit{OSDA}).}                                                                                            \\
        \hline
        \# 3             & \centeritem{Existing methods developed for OSDA attempt to assign smaller weights to target samples of unknown class.
            Despite promising performance achieved by existing methods, the samples of the unknown class are still used for training, which make the model suffer from the risk of negative transfer.}                                                                     \\
        \hline
        \# 4             & \centeritem{Instead of reweighting, this paper presents a novel method namely Thresholded Domain Adversarial Network (\textit{ThDAN}), which progressively selects transferable target samples for distribution alignment. ......} \\
        \bottomrule[0.8pt]
    \end{tabularx}
\end{table*}

